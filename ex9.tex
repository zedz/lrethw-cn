\chapter{Exercise 9: Repetition, Repetition}

Remember in the last two exercises where I kept asking you how you would match
a URL like \verb|/blog/article/12345|? You might have thought the \verb|?| optional
element was the answer, but actually it's repetition.  What you need is a way to
tell the regex engine "any number of these".  There's actually three kinds of
repetition:

\begin{description}
\item[one-or-more] Element repeats such that there has to be at least one.
\item[zero-or-more] Element repeats but it's optional, so there could be 0.
\item[X-or-Y] A minimum of X and a maximum of Y. We'll cover these in the next
    exercise.
\end{description}

The regex symbol for "one-or-more" is \verb|+| and for "zero-or-more" is \verb|*|.
Just like with the "optional" symbol \verb|?| you place it after the regex
symbol or character set to repeat and that's it.

Here's some URLs we'll try matching, with two way to get at a blog article:

\begin{code}{ex9.txt}
\begin{Verbatim}
<< d['code/ex9.txt'] >>
\end{Verbatim}
\end{code}

It's slightly contrived but we'll match by a straight numeric ID or by
the MMM-DD-YYYY date.

\begin{code}{ex9.regex}
\begin{Verbatim}
<< d['code/ex9.regex'] >>
\end{Verbatim}
\end{code}

I took the most complex regex at the end and wrote it out again so you
can see my breakdown of it exactly.

\section{What You Should See}

Running this you should see it match these URLs and you should go back
and match the verbose form to the short form.

\begin{code}{ex9 Output}
\begin{Verbatim}
<< d['code/ex9.regex|regetron']['ex9.txt'] >>
\end{Verbatim}
\end{code}

Best way to map the verbose and short form of this is to go character-by-character
through the regex on line 2, and match the character to the verbose form.

\section{Extra Credit}

\begin{enumerate}
\item Write URLs that do not match these regex at all and explain why they don't.
\item Take the verbose form regex, and edit it so that you convert it back to
    short form manually.
\item Write out the first regex in verbose form.
\item Change the date regex so that it can match "January-01-2011".
\item There's a problem with the second regex because it could match dates like
    "January-34355-929939494".  Is this a good or bad thing?  How would you change
    the regex to make it more strict?
\end{enumerate}

\section{Portability Notes}

Some regex engines confuse \verb|+| and \verb|*|, or simply don't have 
the concept.  If you're trying to use \verb|+| and it's not working, try
just a plain \verb|*| and it should work.  Nearly every regex engine on the
planet has that at least.
