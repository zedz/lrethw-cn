\chapter{Exercise 8: Optional Elements}

Sometimes you want to match something optionally, meaning it can be there or
not and it will still match.  You do this with the \verb|'?'| (question-mark)
character.  Simply put it after the regex symbol or set you want to be optional
and you'll see it match.  Here's a few URLs to try this on:

\begin{code}{ex8.txt}
\begin{Verbatim}
<< d['code/ex8.txt'] >>
\end{Verbatim}
\end{code}

Just a few URLs but imagine you want to match any of those as being the same
thing.  To do that you just put a \verb|'?'| at the end and it will optionally
match the \verb|[0-9]| character set.  I'm going to write one regex but it's
in verbose mode so I can comment on each character and you can see it:

\begin{code}{ex8.regex}
\begin{Verbatim}
<< d['code/ex8.regex'] >>
\end{Verbatim}
\end{code}

First I have the regex like normal, then I wrote it out verbose.  Let's
walk through how the engine would match the first of our regex:

\begin{description}
\item[ex8.regex:3] From the start of \verb|/blog/article/1|...
\item[ex8.regex:4] match \verb|/blog/article/| so that matches.  Now
    we're on the \verb|1| character of the corpus text.
\item[ex8.regex:5] Match a \verb|[0-9]| set, and \verb|1| is in that set
    so match.
\item[ex8.regex:6] The last thing is optional because of \verb|?| but
    it matched so this is matched.
\item[ex8.regex:7] Match the end, and since were are out of characters
    in the corpus text we are done and this matches.
\end{description}

\section{What You Should See}

When you run this you should see it match all of these URLs twice, since
we have the regex repeated in verbose form for you

\begin{code}{ex8 Output}
\begin{Verbatim}
<< d['code/ex8.regex|regetron']['ex8.txt'] >>
\end{Verbatim}
\end{code}


\section{Extra Credit}

\begin{enumerate}
\item Use this process I followed and manually match the regex against the 
    other URLs too.  If it helps, write the regex on a piece of paper, and 
    write the corpus text (the URL) on another part of the paper.  Then underline
    the parts of the URL that match as you walk through the regex and put
    you finger on each regex char that matches.
\item Write some URLs that do not match this regex and explain why they don't.
\item Using more \verb|?| characters, make more of this regex optional.
\item Add 4 optional digits at the end so that you can match an article numbered
    9102.
\end{enumerate}


