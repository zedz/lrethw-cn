\chapter{Exercise 7: The Beginning And End}

I have a list of URLs and I want to match them in certain conditions:

\begin{code}{ex7.txt}
\begin{Verbatim}
<< d['code/ex7.txt'] >>
\end{Verbatim}
\end{code}

I'd like to match these URLs using so with the following routing requirements:

\begin{enumerate}
\item Individual blog articles get matched before the list of blog/articles
    and before the main /blog link.
\item Anything /admin should go somewhere else, but shouldn't get confused
    with /blog URLs.
\end{enumerate}

To do this you use two characters, the \verb|'^'| (caret) and \verb|'$'| (dollar) to
wrap the regex and anchor it at the front and back.  You can also use them separately
to anchor to just one side.

\begin{code}{ex7.regex}
\begin{Verbatim}
<< d['code/ex7.regex'] >>
\end{Verbatim}
\end{code}

In the above I have the same 4 regex listed twice, but the first 4 has the 
anchors added to them so that they only match completely.

\section{What You Should See}

When you run this you can see the effect of the anchors:

\begin{code}{ex7 Output}
\begin{Verbatim}
<< d['code/ex7.regex|regetron']['ex7.txt'] >>
\end{Verbatim}
\end{code}

You can see that the first set works exactly as expected with each regex
matching one URL.  The second set with out the anchors seems to match
all the wrong things because it's finding the element anywhere in the
path.

\section{Extra Credit}

\begin{enumerate}
\item This exercise might have flown past you because it is subtle, so take
    the time to manually enter each of these regex into the Regetron
    shell and see what URLs they match.
\item Rerun this but put !match at the beginning of the script to see if it
    changes things.
\item Add some more URLs and use anchors to match them as well.
\item What happens when you have a URL for article 10?  Why does this fail?
\item Go reasearch the routing engine of a web framework you're familiar with and
    see how they might use anchors or not.
\end{enumerate}

\section{Portability Notes}

Depending on the engine you use this could be a speed improvement or not.
Also some platforms default to this behavior by default, but others do kind
of a half-way version of this.

