\chapter{Exercise 14: Escape Sequences You May Need}

Go to the Python \href{http://docs.python.org/library/re.html}{re library documentation page} and search for "special sequences".  Python's re library is what I'm
using in Regetron and it supports most of the standard special escape sequences 
found in other regex engines.  These sequences can be thought of as "macros"
for common things you do in regular expressions.

\begin{code}{ex14.regex}
\begin{Verbatim}
<< d['code/ex14.regex'] >>
\end{Verbatim}
\end{code}

This script uses the most useful special sequences to match spacing and
"words": \verb|\s|, \verb|\d|, and \verb|\w|.  Go through this script and make
sure you can identify each of the special sequences used and look up 
what they do.

\section{What You Should See}

When you run this script you should see each regex match its line as
expected.

\begin{code}{ex14 Output}
\begin{Verbatim}
<< d['code/ex14.regex|regetron']['ex14.txt'] >>
\end{Verbatim}
\end{code}


\section{Extra Credit}

\begin{enumerate}
\item Rewrite these regex in verbose form and include the names for the special
    sequences used.
\item Come up with your own script that demonstrates the other special sequences.
\item Add the sequences you could get to work to your list of cards and start
    drilling them each night.  To be honest, the only ones I know by heart are
    the ones I used in this exercise's script.
\end{enumerate}

\section{Portability Notes}

Regex engines will typically come up with other escape sequences so you'll
need to refer to your documentation if you're not using the Python re library.
