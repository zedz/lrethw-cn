\chapter{Exercise 16: Search And Replace}

Now that you know about groupings I'm going to show you how to do search and
replace.  If you remember from the last exercise you can surround a regex with
parenthesis and it'll create a capture.  In this exercise I'll show you how
to use that to extract parts of a string and then replace them, as well
as just doing simple replacements.

To do this exercise I'll make a corpus text that has an IP address, some
numbers, and a web URL that's wrong.  Then I'll play with them.

\begin{code}{ex16.txt}
\begin{Verbatim}
<< d['code/ex16.txt'] >>
\end{Verbatim}
\end{code}

I want to make a search/replace that replaces only IP addresses with
"X.X.X.X" so that it is sanitized.  I want another regex that will fix this
URL so that it has \emph{https} and is a correct URL not just a file.

\begin{code}{ex16.regex}
\begin{Verbatim}
<< d['code/ex16.regex'] >>
\end{Verbatim}
\end{code}

The first line I'm doing the IP address replace wrong, which you can see
when we run it as it replace *all* numbers with X.  The second regex is
more correct and replaces only the IP address.  From these two you can see
the form for a replace in Regetron is:

\begin{enumerate}
\item !rep to start a \emph{rep}lace
\item A \verb|/| to set the bound of the \emph{search} part.
\item A regex for searching
\item Another \verb|/| to end the search part.
\item A string for the \emph{replace} part.
\item An ending \verb|/| to end the whole thing.
\end{enumerate}

That means the form is \verb|!rep /SEARCH/REPLACE/| but there's a great modification
to this that comes in handy.  The \verb|/| can be any character, which solves the
problem of doing a search/replace inside a string that has a \verb|/| character
in it already.  I demo this in the third regex by using \verb|!rep ,SEARCH,REPLACE,|
instead.

The final thing to look at is the third regex has a grouping (aka capture), which 
you know about, but then I do \verb|\1|.  What this does it is grabs the first
group that was matched (that's the 1) and \emph{inserts} it right there in the
replace.

\section{What You Should See}

Pay special close attention to this and make sure you understand how the replacements
are working:

\begin{code}{ex16 Output}
\begin{Verbatim}
<< d['code/ex16.regex|regetron']['ex16.txt'] >>
\end{Verbatim}
\end{code}

The most important part is the last regex and the use of the \verb|\1| to grab
the group from the search part and put it in the replace part.  In this case
the \verb|(.+?)| is matching the \verb|index.html| part, and then the \verb|\1|
is placing it in the corrected link inside the replace.  One more thing to
realize is if you had 3 groups in the search, then you'd have access to \verb|\2|
and \verb|\3| as well.

\section{Extra Credit}

\begin{enumerate}
\item If you use Vim or Emacs then you have access to this as a search and 
    replace operation.  In vim try loading the corpus text and typing 
    \verb|:%s ,http://\(.*\)/,https://site.com/\1,|
    which has a slightly different search part so pay attention.  Notice
    I have to escape the parens with \verb|\(|.
\item Write a search replace that replace animals with just "dog", because
    dogs are better.
\item Write a regex that takes a URL with a file path and keeps everything
    but the file, replacing it with \verb|/index.html|.
\item Why did I use \verb|.+?| instead of just \verb|.+| to do the last group?
\end{enumerate}

\section{Portability Notes}

As mentioned in the Extra Credit you have to escape the parenthesis in
groups when you do this in Vim.  Other regex engines use a slightly
different API and way of doing search/replace.

