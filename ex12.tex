\chapter{Exercise 12: Translating Complex Regex}

I'm now going to give you a big assignment to push your learning a bit further.
One way to speed up your education is to periodically pick something seemingly
impossible and sit down to solve it for a few hours.  Even if you completely
fail at it, the act of trying pushes your knowledge past where you thought it
could go.  Sometimes though you figure out that you actually know more than you
thought and you build new confidence.  Doing this also gets you out of a
"training rut" where you just do the same exercises over and over yet feel like
you're never improving.

I've found a reasonably complex regular expression for matching IP addresses.
It has most of the symbols you already know, and some that you don't know.
Here's the regex you'll be working with:

\begin{code}{ex12.regex}
\begin{Verbatim}
<< d['code/ex12.regex'] >>
\end{Verbatim}
\end{code}

What you will do is this:

\begin{enumerate}
\item Write a corpus text file with about 5 IP addresses and 5 dates in it.
\item Make sure this regex matches those 5 IP address and not the dates.  Feel
    free to copy-paste the original regex to make sure you've got it right.
\item Start a copy of the regex and rewrite it in verbose form the way you
    have been.  Remember to put a blank line to start verbose mode, and then
    end it with a blank line.
\item Any symbols you don't know you should look up in the Python
    \href{http://docs.python.org/library/re.html}{re library documentation}.
\item As you rewrite it into verbose form, keep running regetron to make sure
    that the two regex continue to match what they used to match.  Think of
    this as the same as running your unit tests.
\item When you're done, write a single paragraph explaining how this regex
    works.
\end{enumerate}

\section{Extra Credit}

\begin{enumerate}
\item Simplify this regex to a smaller one that matches the same lines.
\item Disprove this new simpler regex is "accurate" by devising a IP address
    that is wrong but which your new regex matches while the bigger one 
    does not.
\item Find another regex of about the same size and convert it as well.
\end{enumerate}


