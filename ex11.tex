\chapter{Exercise 11: This Or That}

You should have been drilling your vocabulary from the last exercise every night and day until you know it cold.  I know you probably absolutely \emph{hate} drilling
and memorizing things, but it's an important skill to pick up if you want to
accelerate your learning on almost any subject.  Memorization isn't useful for
learning how to creatively apply skills, but it is the fastest way to acquire
the foundational basics of a new skill.

Now that you've learned the core of regex solidly I'll show you more advanced things
that work with those building blocks.  First up is the concept of "alternating".
Here's a corpus text we'll work with:

\begin{code}{ex11.txt}
\begin{Verbatim}
<< d['code/ex11.txt'] >>
\end{Verbatim}
\end{code}

What we want to do is match different lines with one regex in 4 different ways:

\begin{code}{ex11.regex}
\begin{Verbatim}
<< d['code/ex11.regex'] >>
\end{Verbatim}
\end{code}

You see the \verb,'|', character between the \verb|[0-9]+| and \verb|[A-Z]+|
expressions? The reason I wanted you to know the basic symbols of regex solidly
is that you will encounter a regex like this and not know what the \verb,|, does.
However, you \emph{do} know what all the other symbols do and can read them.
That means you can break it down and then go find out what the missing pieces mean.
I'll do it for the first one:

\begin{enumerate}
\item from the start
\item 0-9 set
\item one-or-more
\item Hmm, don't know what that is.
\item A-Z set
\item one-or-more
\item to the end
\end{enumerate}

You can now make a guess what the \verb,|, does, probably some kind of "OR" as
in a programming language.  That's actually correct, it says "this OR that expression", and it's called "alternating".  The reason you call this "alternating" is it
causes the regex engine to try both expressions on either side of the regex until
one fails, then it continues until that one matches or they both fail.

\section{What You Should See}

When you run this you'll see that each regex matches a different line or
set of lines:

\begin{code}{ex11 Output}
\begin{Verbatim}
<< d['code/ex11.regex|regetron']['ex11.txt'] >>
\end{Verbatim}
\end{code}

Using the idea of "alternating" try to explain how the line matches and why.

\section{Extra Credit}

\begin{enumerate}
\item Add the \verb,|, "alternator" symbol to your deck of cards and start drilling
    it too.  You can also just call it "OR".
\item Rewrite these regex in verbose mode similar to how I did the breakdown above.
    Just write the English name for the symbol.
\item KEEP DRILLING THOSE SYMBOLS.  It's hard work but you gotta do it, and it 
    pays off big time in making your brain strong.
\end{enumerate}

\section{Portability Notes}

Some regex engines don't implement alternating efficiently and instead add all
sorts of backtracking and other problems.  Be careful when using it and read
the docs to make sure.
